\documentclass{leadsheet-modern}

\begin{document}

\begin{song}[transpose={-2}]{title={Jesus lebt das Grab ist leer},composer={Jörg Streng, Gisela Streng},key={G},capo={1}}

\begin{schedule}
I -- R -- V1 -- R -- V2 -- R -- V3 -- R
\end{schedule}

\begin{chorus}
| ^{G}Je-sus ^{C/G}lebt! \_ Das | ^{D/G}Grab ist ^{G}leer. \_\_ \\
| ^{G}Tod und ^{Em}Grab be- | ^{A}siegt\_ der ^{D}Herr. \_\_ \\
| ^{G}Jesus lebt,\_ ^{C/G}wer | ^{G}ihm ver-traut, \_ \\
| ^{G}der hat auf ^{C}fes - ^{G}ten | ^{C/D}Grund\_ ge- ^{G}baut. \_\_
\end{chorus}

\begin{verse}
Am | ^{D}Kreuz, als Je-sus starb, da schien das | ^{Em}Ende ganz nah, \eighthrest~ \\
doch  | ^{D}zeig-te Gott, dass erst das Kreuz der | ^{G}Wen-de-punkt war! \_ \eighthrest~ \\
Die | ^{Am}Höl-le hat ge-bebt, \_ | ^{D}\_\_ \quarterrest~\eighthrest~ denn | ^{C}Je-sus Chris - tus lebt! | ^{D}\_\_\_\_
\end{verse}

\begin{verse}
Durch seinen Tod zeigt uns der Herr, wie sehr er uns liebt, \\
doch war der Tod nicht Sieger, nein: der Herr hat gesiegt. \\
Des Todes Macht vergeht, denn Jesus Christus lebt!
\end{verse}

\begin{verse}
Weil Jesus einst gesiegt hat und vom Tod auferstand, \\
nimmt auch in unserm Leben nie die Not überhand, \\
und alle Furcht verweht, denn Jesus Christus lebt!
\end{verse}

\end{song}

\end{document}
