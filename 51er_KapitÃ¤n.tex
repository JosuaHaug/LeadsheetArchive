\documentclass{leadsheet} 

\begin{document}

\begin{song}[transpose=5]{title={51er Kapitän},interpret={Reinhard Mey},key={D},tempo={}}

\begin{schedule}
I - V1 - V2 - V3 - V4
\end{schedule}

\begin{intro}
|:^{D}\halfrest~ ^{A}\halfrest~ | ^{G}\wholerest~ :|
\end{intro}

\begin{verse}
Ich ^{D}seh‘ noch, wie mein ^{A}Vater ^{G}aussteigt aus dem ^{D}6 Uhr 20 ^{A}Vororts- ^{G}zug \\
Mit der ^{G}abge- ^{D/F#}wetzten ^{Em}Akten- ^{D}mappe und dem ^{Asus4}grauen Mantel, den er ^{A}trug. \\
Jeden ^{D}Abend stand ich ^{A}da am ^{G}Bahnhof, ich ^{D}war grade ^{A}neun oder ^{G}zehn, \\
Und ich war ^{G}stolz, den ^{D/F#}staub‘gen ^{Em}Siedlungs- ^{D}weg lang ^{Asus4}neben ihm zu ^{A}gehn. \\
Und dann ^{G}mußt‘ er mir jeden ^{A}Abend die immer ^{D}gleiche Geschichte erzähl‘ ^{G}n: \\
Wie ^{Em}einmal alles mit uns werden ^{A}würde, und da ^{Em}durfte kein Wort fehl‘^{A}n. \\
Und ^{G}immer vor der Haustür mußt‘ er ^{A}sagen: \frqq Eines ^{D}Tages ^{D/A}wirst du ^{G}sehn. \\
Da werden ^{Em}wir beide hier vor-  ^{A}fahr‘n in einem ^{Em}schneeweißen ^{A}51er Kapitän! \flqq \\
\end{verse}

\begin{verse}
Und die Sitze sind aus rotem Leder und der Himmel ist wie ein Dom, \\
Und der Lack glänzt in der Sonne, und überall funkelt Chrom. \\
Die Motorhaube nimmt kein Ende und die Kühlerfigur blitzt, \\
Und du glaubst, du würdest schweben, wenn du hinterm Lenkrad sitzt!“ \\
Ich hatte eine schwarze Trainingshose und mein Vater besaß ein Paar Schuh‘, \\
Aber wenn er so erzählte, dann fehlte nicht mehr viel dazu, \\
Und wenn ich meine Augen schloß, dann konnte ich uns wirklich sehn: \\
Meinen Vater und mich vor der Haustür in einem schneeweißen 51er Kapitän! \\
\end{verse}

\begin{verse}
Nun, es kamen andre Zeiten, es ging voran und irgendwann \\
Kam mein Vater dann tatsächlich eines Abends mit einem Auto an: \\
Es war ein steinalter Olympia, bei dem immer der Gaszug riß, \\
Bei dem nie die Heizung ausging, im Grunde war das ein Totalbeschiß. \\
Meinen Vater aber war er gut genug, das war genau seine Art, \\
An sich selber immer rumzuknausern, an uns hat er nie gespart. \\
Mir jede Chance im Leben geben, mich einmal auf dem Treppchen zu sehn, \\
Das war es: der totale Luxus war sein schneeweißer 51er Kapitän. \\
\end{verse}

\begin{verse}
Er hat nie mehr davon gesprochen, doch ich weiß, er hat davon geträumt. \\
Vielleicht war das so ein Symbol für eine Chance, die man versäumt. \\
Heut würd‘ ich ihm gern einen schenken, ich weiß sogar, wo einer steht: \\
Rotes Leder, Weißwandreifen, und sogar das Radio geht. \\
Mein Vater ist vor ein paar Jahr‘n gestorben, es hat nicht hingehau‘n diesmal, \\
Nicht einmal auf seiner letzten Fahrt, da war‘s ein schwarzer Admiral. \\
Aber wenn es einen Himmel geben sollte, dann werd‘ ich ihn endlich sehn: \\
Denn dann holt mein Alter Herr mich ab in einem schneeweißen 51er Kapitän! \\
\end{verse}

\end{song}
\end{document}