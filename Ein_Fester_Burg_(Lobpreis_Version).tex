\documentclass{leadsheet-modern}

\begin{document}

\begin{song}[remember-chords]{title={Ein Feste Burg ist unser
Gott},interpret={Lobpreisband Eberdingen},lyrics={Martin Luther},composer={Swen Seemann, Josua
Haug},key={E},tempo={66 bpm}}

\begin{schedule}
\songpart{I}
\songpart{V1}
\songpart{V2}
\songpart{V3}
\songpart{Vinst}
\songpart{V4}
\songpart{I}
\end{schedule}

\begin{intro}
|: ^{C#m}\wholerest~ |^{C#m/H}\wholerest~ |^{C#m/H}\wholerest~
|^{C#m/A}\wholerest~ |^{H9}\wholerest~
\end{intro}

\begin{verse}
Ein |^{C#m}feste Burg ist |^{C#m/H}unser Gott, ein |^{C#m/A}gute Wehr und
|^{H9}Waffen.
\\
Er |^{C#m}hilft uns frei aus |^{C#m/H}aller Not, die |^{C#m/A}uns jetzt hat
be-|^{H}troffen.
\\
Der |^{A}alt böse ^{H}Feind, mit |^{C#m}Ernst er's jetzt ^{E}meint, |^{F#m}groß
Macht und viel |^{G#sus4}List ^{G#}
\\
sein |^{A}grausam ^{H}Rüstung |^{C#m}ist, auf |^{A}Erd ist nicht
seins|^{H}gleichen.
\end{verse}

\begin{verse}
Mit |^unsrer Macht ist|^nichts getan, wir |^sind gar bald ver-|^loren; \\
es |^streit' für uns der |^rechte Mann, den |^Gott hat selbst er-|^koren. \\
Fragst |^du, wer der ^ist? Er |^heißt Jesus ^Christ, |^der Herr Zeb|^oth, ^~ \\
und |^ist kein andrer |^Gott; das |^Feld muss er be-|^halten.
\end{verse}

\begin{verse}
Und |^wenn die Welt voll |^Teufel war und |^wollt uns gar ver-|^schlingen, \\
so |^fürchten wir uns |^nicht so sehr, es |^soll uns doch ge-|^lingen. \\
Der |^Fürst dieser ^Welt, wie |^saur er sich ^stellt, |^tut er uns doch ^nicht;
^~ \\
das |^macht, er ^ist ge-|^richt': ein ^Wörtlein kann ihn |^fällen.
\end{verse}

\begin{verse}
Das |^Wort sie sollen |^lassen stahn und |^kein Dank dazu |^haben; \\
er |^ist bei uns wohl |^auf dem Plan mit |^seinem Geist und |^Gaben. \\
Nehm-|^en sie den ^Leib, Gut, |^Ehr, Kind und ^Weib: |^lass fahren ^dahin, ^~
\\
sie|^habens ^kein Ge-|^winn, das |^Reich muss uns doch bleib-|^en.
\end{verse}

\end{song}
\end{document}