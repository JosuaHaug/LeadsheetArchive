\documentclass{leadsheet-modern}

\begin{document}

\begin{song}{title={Ich steh an deiner Krippe hier},composer={Johann Sebastian Bach, Johannes Zahn},lyrics={Paul Gerhardt},key={Am}}

\begin{schedule}
\songpart{I}
\songpart{V1}
\songpart{V2}
\songpart{V3}
\songpart{V4}
\songpart{V5}
\songpart{V6}
\songpart{V7}
\songpart{V8}
\songpart{V9}
\end{schedule}

\begin{intro}
|^{Am}\wholerest~ ||
\end{intro}

\begin{verse}
Ich |^{Am}steh an dei- ner |^{G}Krip - pe ^{C}hier, 
o |^{Am}Je - su, ^{F}du mein |^{Am/E}Leb - ^{E}en; \\
ich |^{Am}kom - me, bring und |^{G}schen - ke ^{C}dir, 
was |^{Am}du mir ^{F}hast ge - |^{Am/E}ge - ^{E}ben. \\
Nimm |^{C}hin, es ^{F}ist mein |^{G}Geist und ^{C}Sinn, 
Herz, |^{C}Seel und ^{F}Mut, nimm |^{G}al - les ^{C}hin \\
und |^{D}lass ^{E/D}dirs ^{Am/C}wohl\_\_ ^{Dm7}ge - |^{E}fal - |^{Am}len. 
\end{verse}

\begin{verse}
Da ich noch nicht geboren war, 
da bist du mir geboren \\
und hast mich dir zu Eigen gar, 
eh ich dich kannt, erkoren. \\
Eh ich durch deine Hand gemacht, 
da hast du schon bei dir bedacht, \\
wie du mein wolltest werden. 
\end{verse}

\begin{verse}
Ich lag in tiefster Todesnacht, 
du warest meine Sonne, \\
die Sonne, die mir zugebracht 
Licht, Leben, Freud und Wonne. \\
O Sonne, die das werte Licht 
des Glaubens in mir zugericht', \\
wie schön sind deine Strahlen! 
\end{verse}

\begin{verse}
Ich sehe dich mit Freuden an 
und kann mich nicht satt sehen; \\
und weil ich nun nichts weiter kann, 
bleib ich anbetend stehen. \\
O dass mein Sinn ein Abgrund wär 
und meine Seel ein weites Meer, \\
dass ich dich möchte fassen! 
\end{verse}

\begin{verse}
Wann oft mein Herz im Leibe weint 
und keinen Trost kann finden, \\
rufst du mir zu: „Ich bin dein Freund, 
ein Tilger deiner Sünden. \\
Was trauerst du, o Bruder mein? 
Du sollst ja guter Dinge sein, \\
ich zahle deine Schulden.“ 
\end{verse}

\begin{verse}
O dass doch so ein lieber Stern 
soll in der Krippen liegen! \\
Für edle Kinder großer Herrn 
gehören güldne Wiegen. \\
Ach Heu und Stroh ist viel zu schlecht, 
Samt, Seide, Purpur wären recht, \\
dies Kindlein draufzulegen! 
\end{verse}

\begin{verse}
Nehmt weg das Stroh, nehmt weg das Heu, 
ich will mir Blumen holen, \\
dass meines Heilands Lager sei 
auf lieblichen Violen; \\
mit Rosen, Nelken, Rosmarin 
aus schönen Gärten will ich ihn \\
von oben her bestreuen. 
\end{verse}

\begin{verse}
Du fragest nicht nach Lust der Welt 
noch nach des Leibes Freuden; \\
du hast dich bei uns eingestellt, 
an unsrer statt zu leiden, \\
suchst meiner Seele Herrlichkeit 
durch Elend und Armseligkeit; \\
das will ich dir nicht wehren. 
\end{verse}

\begin{verse}
Eins aber, hoff ich, wirst du mir, 
mein Heiland, nicht versagen: \\
dass ich dich möge für und für 
in, meinem Herzen tragen. \\
So lass mich doch dein Kripplein sein; 
komm, komm und kehre bei mir ein \\
mit allen deinen Freuden! 
\end{verse}

\end{song}

\end{document}

