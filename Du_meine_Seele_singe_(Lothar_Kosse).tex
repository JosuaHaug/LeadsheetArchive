\documentclass{leadsheet-modern}

\begin{document}

\begin{song}[remember-chords,transpose=7]{title={Du meine Seele sing},interpret={Lothar Kosse},composer={Paul Gerhardt},key={A},tempo={124}}

\begin{verse}
^ADu meine Seele, ^Esinge, \\
wohl^{A/C#}auf und ^Esinge ^Aschön \\
^ADem, welchem alle ^EDinge \\
zu ^{A/C#}Dienst und ^EWillen ^Astehn. \\
^AIch will den ^DHerren ^{A/C#}droben \\
hier ^{Bm}preisen ^{C#}auf der ^{F#m}Erd; \\
^{E/G#}Ich ^Awill ihn herzlich ^Dlob- ^{A/C#}en, \\
so- ^{F#m7}lang ich ^Eleben ^Awerd.
\end{verse}

\begin{verse}
^Wohl dem, der einzig ^schauet \\
nach ^Jakobs ^Gott und ^Heil! \\
^Wer dem sich anver^trauet, \\
der ^hat das ^beste ^Teil, \\
^Das höchste ^Gut er^lesen, \\
den ^schönsten ^Schatz ge^liebt; \\
^Sein ^Herz und ganzes ^We- ^sen \\
bleibt ^ewig ^unbe- ^trübt.
\end{verse}

\begin{verse}
Hier sind die starken Kräfte, / die unerschöpfte Macht; /
Das weisen die Geschäfte, / die seine Hand gemacht: /
Der Himmel und die Erde / mit ihrem ganzen Heer, /
Der Fisch unzähl'ge Herde / im großen wilden Meer.
\end{verse}

\begin{verse}Hier sind die treuen Sinnen, / die niemand Unrecht tun, /
All denen Gutes gönnen, / die in der Treu beruhn. /
Gott hält sein Wort mit Freuden, / und was er spricht, geschicht; / 
Und wer Gewalt muss leiden, / den schützt er im Gericht.
\end{verse}

\begin{verse}Er weiß viel tausend Weisen, / zu retten aus dem Tod, /
Ernährt und gibet Speisen / zur Zeit der Hungersnot, /
Macht schöne rote Wangen / oft bei geringem Mahl; /
Und die da sind gefangen, / die reißt er aus der Qual.
\end{verse}

\begin{verse}
^Er ist das Licht der ^Blinden, \\
er^leuchtet ^ihr Ge^sicht, \\
^Und die sich schwach be^finden, \\
die ^stellt er ^aufge- ^richt'. \\
^Er liebet ^alle ^Frommen, \\
und ^die ihm ^günstig ^sind, \\
^Die ^finden, wenn sie ^kom- ^men, \\
an ^ihm den ^besten ^Freund.
\end{verse}

\begin{verse}
Er ist der Fremden Hütte, / die Waisen nimmt er an, /
Erfüllt der Witwen Bitte, / wird selbst ihr Trost und Mann. /
Die aber, die ihn hassen, / bezahlet er mit Grimm, /
Ihr Haus und wo sie saßen, / das wirft er um und um.
\end{verse}

\begin{verse}
Ach ich bin viel zu wenig, / zu rühmen seinen Ruhm; /
Der Herr allein ist König, / ich eine welke Blum. /
Jedoch weil ich gehöre / gen Zion in sein Zelt, /
Ist's billig, dass ich mehre, / sein Lob vor aller Welt.
\end{verse}

\end{song}
\end{document}
